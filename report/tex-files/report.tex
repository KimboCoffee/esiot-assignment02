\documentclass[a4paper,12pt]{report}

\usepackage{alltt, fancyvrb, url}
\usepackage{graphicx}
\usepackage[utf8]{inputenc}
\usepackage{float}
\usepackage{hyperref}
\usepackage{cleveref}

\title{Embedded Systems and Internet-of-Things \\ - \\ Second Assignment}

\author{Kimi Osti}
\date{\today}


\begin{document}
	
	\maketitle
	
	\begin{abstract}
		The main goal of this project is to realize a Smart Liquid Waste Disposal Container. It will be based on an Arduino UNO Board linked to a computer via serial communication. In this system, some hardware devices will be dedicated to user interaction, opening or closing the container, or exchanging some basic information via a screen. Then, a subsystem will be dedicated to waste monitoring, revealing when the container is too full or its content is too hot, blocking user interaction. To solve these problems, an operator will have to intervene via the computer to empty the container or to signal that the temperature situation is now under control.
	\end{abstract}
	
	\tableofcontents
	
	\chapter{Domain Analysis}
	The domain analysis, working on an embedded system, can be conducted on two levels: hardware and software.
	
	\section{Hardware}
	Besides the main container, which is by no means an active hardware device, some main components can be highlighted from a functional standpoint:
	
	\subsection{User Detector}
	The user detector has the main function of revealing when a user is nearby. This is needed to wake the system from low power consumption mode when a potential user approaches the system.\newline
	It is realized with an infrared passive sensor (PIR), whose functioning is based on infrared variations in the environment. These devices, powered with a tension, return a digital output on a pin when they detect abrupt infrared variations. In \href{https://learn.adafruit.com/pir-passive-infrared-proximity-motion-sensor/}{the variant used} for this project, there are two knobs on the back to regulate sensitivity and delay time, two key factors to detect changes. Due to the fact that these devices work on variations of infrared instead of absolute values, a calibration time on system startup might have to be taken into account.
	
	\subsection{Waste Level Detector}
	The waste level detector has the vital function of detecting the amount of waste introduced in the container. This has to be monitored so that a maximum level is never exceeded.\newline
	It can be realized thanks to a proximity sensor, which detects how close the level of waste is to the top of the container. When this distance goes below a certain defined threshold, it means the container is too full. For our project, an \href{https://docs.google.com/document/d/1Y-yZnNhMYy7rwhAgyL_pfa39RsB-x2qR4vP8saG73rE/edit?tab=t.0}{ultrasound proximity sensor} is used rather than a light proximity sensor. This because it allows greater precision on shorter distances (because of how much sound is slower than light) and because it is less sensitive of orientation of the measured object. This device is powered by a trigger impulse, which produces a sound that is captured back and produces an echo impulse. Measuring the time elapsed between these two signals, and knowing - even approximately - the speed of sound through air a measure of distance can be obtained.
	
	\subsection{Door}
	The container's door is realized here thanks to an electrical motor, which is oriented to a certain angle to open or close the container.\newline
	Rather than using DC motors, in which the angle can't be specified, or stepper motors, that offer a fine grained rotating precision but are very expensive, we have chosen to use \href{https://www.kjell.com/globalassets/mediaassets/701916_87897_datasheet_en.pdf?ref=4287817A7A}{a servo motor}. It has a range of about 180°, which is enough if applied to a door's control, and it offers the possibility to control the angle of its rotation in an easy way. For it to work properly, a servo motor requires that electrical pulses of a certain duration (corresponding to the angle) are sent to it at a fixed frequency, so that it is oriented to meet the desired angle. Particular attention has to be paid to how much the door resists to rotation because electrical consumption depends on this, so having a door that weights too much can damage the UNO Board because of current consumption.
	
	\subsection{Open and Close buttons}
	These are two simple tactile buttons linked to the UNO Board.\newline
	Their functioning is based on the fact that they receive power from the board, and transfer it back to it via some pin when they are pressed. In order to have 0 rather than VCC as the default value, a pull down resistor - 10KOhm in our case - is needed to connect the button to the ground.
	
	\subsection{Red and Green LEDs}
	These are two LEDs used to signal to the user the state of the system: green is turned on when the container is in a ready-to-use state, while red signals some problem that blocks the system functions.\newline
	In order for the LEDs to work, tension has to be applied on their two ends: the anode is connected to a pin powering it based on the desired output value, and the cathode is connected to the ground. Since the LEDs used for this project require a tension of 2V to work properly, and Arduino UNO's VCC is 5V, a 220Ohm resistor is needed in order to lower the tension at the ends of each LED.
	
	\subsection{User Screen}
	This device is used to show the user some simple information, such as the fact that the system is ready to accept waste and how to open it.\newline
	In our system, a \href{https://robot-italy.com/products/16x2-lcd-display-green?_pos=3&_psq=display+lcd&_ss=e&_v=1.0}{green background Liquid Crystal Display} is used for this function. More specifically, it's used with a \href{https://en.wikipedia.org/wiki/I%C2%B2C}{I2C} adapter, to exploit its features occupying less pins on the board.
	
	\subsection{Temperature Sensor}
	This device is needed to monitor the waste temperature.\newline
	It can be realized using a simple analog temperature sensor. These devices are based on the fact that temperature increases or decreases the electricity flow through them, and an analog signal is returned to the board according to the temperature value. Starting from this analog value, knowing voltage and temperature ranges of the devices (which have to be read on technical data sheets) the measure can be easily obtained via some mathematical formula. In this case, a \href{https://www.analog.com/en/products/tmp36.html?doc=TMP35_36_37.pdf#part-details}{TMP36 temperature sensor} was used.
	
	\subsection{Operator Dashboard}
	The operator dashboard is meant to be accessed only by operators, and it has to be connected to the Arduino UNO Board.\newline
	In our system, it will be hosted on a separate general purpose computer, in which a program will run to handle the dashboard's features. Connection will be achieved via serial communication, exploiting the USB connection offered by the Arduino UNO Board.
	
	\begin{figure}[H]
		\centering{}
		\includegraphics[scale=0.50]{img/HW-domain.png}
		\caption{Schematic disposition of hardware devices, from the \href{https://docs.google.com/document/d/1iFXGmo7RVZMpJ5bxUN5ms_qFqg2B-wecRc0sfas9rQ4/edit?usp=sharing}{professor's assignment}}
		\label{img:hw-domain}
	\end{figure}
	
	\section{Software}
	Software domain analysis can be divided in two main modules, one running on the Arduino UNO Board (the Smart Waste Disposal Control Unit) and the other running on the computer connected to it (the Operator Dashboard Application).
	
	\subsection{Smart Waste Disposal Control Unit}\label{subsec:system-control-unit}
	This is the core unit of the device, being the one that directly controls and regulates the behavior of the entire system.\newline
	The main constraint in terms of technologies is that it has to be programmed in C++ with the Wiring framework, since it will run on an Arduino UNO Board. Trying to dissect the system's behavior, it can easily be decomposed in three main Tasks - here represented in their behavior exploiting state charts - that run in parallel.
	
	\subsubsection{Temperature Monitor Task}
	The temperature monitor behaves regardless of the state in which the other parts of the systems are, signaling whenever an exception happens.
	\begin{figure}[H]
		\centering{}
		\includegraphics[scale=0.60]{img/temp-monitor-FSM.png}
		\caption{FSM for the temperature monitor}
		\label{img:temp-monitor-FSM}
	\end{figure}
	In this scheme, \emph{Working} should be interpreted as any state of the main Task, to which a signal is sent when entering the \emph{NonWorking} state. In this way, the main Task can regulate its own behavior according to the possible exceptions generated by this task. 
	
	\subsubsection{Level of Waste Monitor Task}
	This Task, similarly to the previous one, can be modeled as an independent control flow that can signal exceptions to the main Task. 
	\begin{figure}[H]
		\centering{}
		\includegraphics[scale=0.60]{img/waste-level-monitor-FSM.png}
		\caption{FSM for the waste level monitor}
		\label{img:waste-level-FSM}
	\end{figure}
	Like in the previous scheme, \emph{Working} refers to any state of the main Task, and \emph{NonWorking} refers to a generic state in which an exception is received and the normal functions of the system are suspended. Also for this feature, signals between distinct Tasks regulate the system's overall behavior.
	
	\subsubsection{Container Handler Task}
	The main life cycle of the system can be described as a separate Task.
	\begin{figure}[H]
		\centering{}
		\includegraphics[scale=0.60]{img/main-FSM.png}
		\caption{FSM for the main system life cycle}
		\label{img:main-FSM}
	\end{figure}
	In this Task, \emph{SensorException} is any kind of signal that is sent from any of the Tasks seen before - which are depending on sensor measures - to the system in order to stop its default life cycle. On the other hand, \emph{DashboardProblemSolved} is the corresponding dashboard operation that restores the system's functions.
	
	\subsection{Operator Dashboard Application}
	The Operator Dashboard Application is a software module running on a general purpose PC. It's supposed to show a Graphical User Interface to the Operator, a qualified user that can restore the normal functioning of the container, emptying it or resetting it after a critical temperature situation is solved.\newline
	This project component must feature these two functions, plus optionally the ability to inspect some data history, but is on any other aspect - such as programming language or software architecture - freely developed by the programmer, who will chose its preferred technologies.
	
	\chapter{Development}
	The development solutions can be split in three main areas: Hardware Development, regarding the circuit and its realization, Embedded System - regarding the development of the software running on the Arduino board, and Computer System, tackling the main aspects between the Operator Dashboard's implementation.
	
	\section{Hardware Development}
	\begin{figure}[H]
		\centering
		\includegraphics[width=\textwidth]{img/circuit-detail.png}
		\caption{\href{https://www.tinkercad.com/things/1OaxE8QuWbw-esiot-assignment02?sharecode=9QQajaw8It3u4rzvy1XqA17RKAIlDJlUtWD6_dBJsmI}{Full circuit in detail}}
		\label{img:full-circuit}
	\end{figure}
	When building the circuit, some aspects regarding specific pin choices has to be made. This because some pins are fixed (such as A4 for SDA and A5 for SCL in I2C communication), or because some devices require specific features (such as interrupts for the user detector). These aspects are discussed in further detail in the software Development analysis.
	\newline Furthermore, having so many devices connected to the board, which offers a single 5V output line, means that power consumption has to be taken into account. For example, the servomotor, which requires a lot of electricity to work, has to be detached whenever it's not in use.
	
	\section{Computer System}
	The computer system is here implemented in Python, since it allows to simply create user interfaces and serial communication.\newline
	Particularly, \href{https://docs.python.org/3/library/tkinter.html}{the tkinter interface package} is used for producing the user interface. This package simplifies the development process behind GUIs offering a high-level interface for the Tcl/Tk GUI Toolkit.\newline
	Serial communication, on the other hand, is achieved thanks to \href{https://pyserial.readthedocs.io/en/latest/}{the serial module of the pyserial library}. This module tackles the low-level parts of the serial communication implementation, and allows the programmer to deal with higher level matters.\newline
	From an implementing standpoint, the System consists on three main modules: a Message Handler, the App and the View. The App controls the application's flow, updating each period the View to correctly represent the data read by the Message Handler.
	\paragraph{Communication protocol} To ensure effective communication, a simple protocol is established between the two systems.
	\newline From Arduino towards the computer, each message contains 5 lines. They represent respectively the current scheduler period, the measured temperature, the eventual occurring of a temperature exception, the current waste level and the eventual occurring of a level exception.
	\newline From the computer towards Arduino, communication is much simpler. Since it has just two possible values to signal, a byte is enough. In particular, 1 is coded to represent the temperature restoring, and 2 is coded to represent the emptying command.
	\newline Synchronization between the two systems could have been problematic, but Arduino UNO automatically reboots whenever a new device is connected to its serial line. This means that the computer is sure to be reading real-time data produced by the board, rather than some old values still saved in a buffer. Anyways, the serial line is cleared at system startup so to be sure that there are no old values stored in memory that could affect the reads.
	
	\section{Embedded System}
	The software for the Embedded System is developed with the \href{https://wiring.org.co/}{Wiring framework}, which is based on C++ and allows the programmer to exploit all the features of that language, adding some high level interfaces for hardware interaction purposes. This framework, based on a super loop architecture, very well fits Embedded Systems programming, since they require to run non-stop. Furthermore, being based on C++, it allows the programmer to exploit Object Oriented Programming, that very effectively represents this kind of systems, in which different devices work together and exchange messages.
	
	\subsection{Implementing Solutions}
	Implementing solutions can be divided in two main groups, depending on whether they concern mainly device aspects or system regulation matters.
	
	\subsubsection{System Regulation Implementing Choices}
	The system is realized as a Synchronous Finite State Machine. This means that the whole system runs at a fixed period. At the end of each period, guard conditions are evaluated to check whether a state transition has to be triggered.
	\newline It was previously noted - \ref{subsec:system-control-unit} - how the system can be divided in three Tasks. This means that every one of the Tasks has its own period, and the Finite State Machine has to run at a period such that all Tasks can run at their own. This would theoretically mean running at a period that equals the Greatest Common Divisor between Tasks' periods, but in practice this could cause problems because it might result in a period that is too short. Having a period which is too short means that not all the Tasks can be executed in a single period (there certainly is a period in which all Tasks are executed, since the FSM period is an exact divisor of all Task periods). On the other hand, introducing a period which is not an exact divisor of every Task's period would mean that some Tasks don't run their step at their exact period (introducing in practice some jitter), but may prevent period overruns.
	\newline To assure control in Tasks execution, a basic Scheduler has to be implemented. Particularly, a Round Robin Cooperative Scheduler was implemented. This means that each Task must have a finite execution step - since a Cooperative Scheduler requires the Task itself to free the CPU, opposing to a Preemptive Scheduler which can regain CPU control via interrupts - and that some implicit priorities are set between Tasks, depending on their execution order in a single Scheduler tick. This was exploited executing the Monitor Tasks before the Container Handler, in order to record possible exceptions in the least possible time.
	\newline To correctly setup the system's behavior, Tasks also have to communicate between them. To ensure inter-Task communication, a shared object tracking the System overall state (working o non-working) was used. On this object, the two Monitor Tasks could set the values which had to be read by the Container Handler.
	\newline On the other hand, in order to communicate on the serial line, a dedicated Task was implemented. This task is the only one accessing the serial line, and keeps a pointer to the two Monitor task in order to read their measures and signal eventual Dashboard restoring actions.
	\newline From an implementing standpoint, the Scheduler and the Container Handler Task were set to run at a period of 50ms, while the monitors were set to produce 5 measures per second. The communication Task, which works on the data produced by the Monitor Tasks, was logically set to run at their same period. Since the system was developed to work as a synchronous Finite State Machine, interrupts were avoided when possible. Also for time measuring matters, a simple wrapper class named TimerImpl was implemented. It exploits the \href{https://github.com/sstaub/Timer}{Timer library}, which isn't interrupt based.
	
	\subsubsection{Device Implementing Choices}
	Each device was represented in this system as an object. This allowed maximum flexibility in the Development phase, and would also allow some grade of scalability (for example, a temperature monitor task might use more than one sensor to produce a more reliable measure).
	\newline As underlined before, interrupts weren't exploited in this system except from cases in which they were necessary. One of these cases was the PIR User Detector, which exploits interrupt to wake up the system when in deep sleep state. This is the main reason why it was connected to digital pin 2, that supports interrupts by default. The interrupt handler is an empty routine, so not to waste unnecessary time nor resources, and is detached from the pin as soon as the system wakes back up.
	\newline To implement the LCD Screen behavior, the \href{https://docs.arduino.cc/libraries/liquidcrystal-i2c/}{LiquidCrystal I2C library} was used. This library relies on I2C protocol to allow communication between the Arduino board and a LCD screen, providing a wrapper class to handle all its features without dealing with low level aspects.
	\newline The door is implemented relying on the \href{https://github.com/nabontra/ServoTimer2}{Servo Timer 2} library, which exposes a wrapper class to handle servo motors through timer 2, which is 8bit, leaving timer 1 - the only 16bit timer in Arduino UNO - free for other purposes. One of the most critical aspects in this project was motor detachment: delaying the call whenever it's set might lead to period overruns, while not detaching the motor might lead to electrical consumption issues. The solution adopted here relies on the fact that the method call to set an angle on the motor is non-blocking: considered this, and that a simple counter can track the time elapsed from the last motor usage, the scheduler and other Tasks keep running, and 200ms (a threshold defined by the developer) after any motor action the motor itself was detached.
	\newline The sonar device for filling measurement was programmed to calibrate thanks to a starting measure, which is taken as reference for the empty container. This is done to improve portability through different possible environments. It also considers air temperature to be constant at 25°C, not relying on the temperature sensor measures because they are linked to the Liquid Waste temperature rather than air temperature.
	
	\chapter{System Demonstration}
	\href{https://drive.google.com/file/d/1WgMCrVB_kaROzejK9e0wBk0y-nm3afkx/view?usp=sharing}{This is a simple demonstration of the system.}
	
\end{document}
